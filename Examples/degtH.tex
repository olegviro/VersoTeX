\documentclass{article}
\usepackage{verbatim,amssymb,amsmath,array}
\usepackage{vo}
\pagestyle{empty}

\begin{document}
\title{
Stiefel orientations\\
on a real algebraic variety}
\author{A. I. Degtyarev}
\maketitle
\abstract{
Some natural Stiefel orientations on the normal bundle of the fixed point
set of an involution on a smooth manifold are constructed. This result
is applied to non-singular real algebraic varieties in order to generalize
Rokhlin's construction of complex orientations on a separating real
algebraic curve}

\keywords{
Stiefel orientation, involution, fixed point set, real algebraic variety}

\hide{$\def\le{\leqslant}
\def\ge{\geqslant}
\def\longto{\longrightarrow}
\def\into{\hookrightarrow}
\def\first{\roster\runinitem}
\def\o{\otimes}
\def\s{\#}
\def\tm{\times}
\def\pt{\text{\mathrm{pt}}}
\def\i{\operatorname{in}}
\def\pr{\operatorname{pr}}
\def\const{\operatorname{const}}
\def\id{\operatorname{id}}
\def\inc{\operatorname{inclusion}}
\def\conj{\operatorname{conj}}
\def\tconj{\widetilde{\conj}}
\def\rel{\operatorname{rel}}
\def\Ann{\operatorname{Ann}}
\def\Im{\operatorname{Im}}
\def\Fl{\operatorname{Flag}}
\def\St{\operatorname{St}}
\def\Hom{\operatorname{Hom}}
\def\Ker{\operatorname{Ker}}
\def\Fix{\operatorname{Fix}}
\def\F{\mathcal F}
\def\CS{\mathcal S}
\def\SF{\CS\F}
\def\SH{SH}
\def\Sm{Sm}
\def\sm{\operatorname{sm}}
\def\Pin{\mathrm{Pin}}
\def\Spin{\mathrm{Spin}}
\def\Z{\mathbb Z_2}
\def\R{\mathbb R}
\def\C{\mathbb C}
\def\Rp{\R p}
\def\Cp{\C p}
\def\RA{\R A}
\def\CA{\C A}
\def\Rh{\R h}
\def\Ch{\C h}
\def\tR{\widetilde{\RA}}
\def\tC{\widetilde{\CA}}
\def\pR{\pr_\R}
\def\pC{\pr_\C}
\def\d{\partial}
\def\cd{\delta}
\def\k{\varkappa}
\def\t{\theta}
\def\f{\varphi}
\def\sg{\sigma}
\def\Sg{\Sigma}
\def\L{\Lambda}
\def\be{{\pmb e}}
\def\E#1/#2/#3/{{}^{#1}\!E^{#2}_{#3}}$}
%
\spnewtheorem{proclaim}{}[subsection]{\bf}{\it}
\renewcommand{\theproclaim}{\arabic{section}.\arabic{subsection}.\arabic{proclaim}}
\spnewtheorem{remark}{Remark}[proclaim]{\bf}{\rm}
\spnewtheorem{definition}{Definition}[proclaim]{\bf}{\rm}
%
\setcounter{section}{-1}
\section{ Introduction }
\subsection*{ 0.1 }
The main purpose of this paper is to generalize Rokhlin's construction
of complex orientations on a separating real algebraic curve.
The original construction is the following (Rokhlin \cite{6}): 
Denote, respectively, by $\CA$ and $\RA$ the sets of real and complex
points of curve $A$. Since $\RA$ separates $\CA$, the complement
$\CA\setminus\RA$ consists of two components $\CA_{\pm}$. The complex
orientation of complex manifolds $\CA_{\pm}$ defines a pair of
opposite orientations of their common boundary $\RA$.

A direct generalization of this construction is obvious: Let $F$ be
an $m$-dimensional smooth submanifold of a smooth $n$-dimensional
manifold $X$. Suppose that the fundamental class $[F]$ of $F$ vanishes
in $H_m(X)$. Then the linking coefficient map
$\operatorname{lk}:H_{n-m-1}(X\setminus F)\longto\Z$ restricted to the
boundary of a tubular neighborhood of $F$ in $X$ is an
$(n-m-1)$-dimensional Stiefel orientation on the normal bundle
$\nu_F$ of $F$ in $X$. (The notion of Stiefel orientation is a
generalization of the ordinary orientation and $\Spin$-structure on
a vector bundle; see Definition 1.4.1 for details.) This orientation
is well defined over the subgroup 
$\Ker[\inc_*:H_{n-m-1}(F)\to H_{n-m-1}(X)]$. O.Viro \cite{8} proposed
a further generalization of this construction (see 4.2) for the case
when pair  $(X,F)$  is  supplied  with  the  following  additional 
structure:
$X$ is a manifold with an involution $c:X\to X$, and $F=\Fix c$ is
the fixed point set of $c$. In this paper we give another approach
to Viro's construction which shows that the arising Stiefel orientations
are, in fact, also the linking coefficient functionals on homology
groups of some appropriate spaces $\CS^{k-1}X\setminus \CS^{k-1}F$
associated with pair $(X,c)$. These orientations exist if for some
$k\ge1$ class $[F]$ vanishes under some ``higher'' inclusion maps
$e_{m+k-1}^k:H_*(F)\dashrightarrow H_{m+k-1}(X)$, and it is defined over
the subgroup $\Ker e_{n-m-1}^k$ (see Theorem 4.1.1 for details).
\begin{remark}{{Remark} The advantage of the approach of this paper is
that the construction is connected with the general theory of
$\Z$-spaces (i.e. spaces with involution). In particular, if $F$
is an $M$-space (i.e. $\dim H_*(F)=\dim H_*(X)$), this theory enables
to express the condition necessary for existence of the orientations
in terms of characteristic classes of $X$ (Degtyarev \cite{3}).
}\end{remark}
\subsection*{ 0.2 }
Applying Theorem 4.1.1 to an $n$-dimensional real algebraic manifold $A$
yields a series of partial $i$-dimensional Stiefel orientations on the
tangent bundle of $\RA$, \ $n-k\le i\le n-1$, provided that
$e_{n+k-1}^k[\RA]=0$. However, this condition seems to be rather 
difficult to be verified. If $A$ is a projective complete intersection,
it can be substantially simplified and reduced to the usual requirement
that $[\RA]$ should be either equal to zero or dual to the plane section
class in $H_n(\CA)$. In this case some modified construction provides a
series of Stiefel orientations of all dimensions on the tangent bundle
of the double cover $\tR$ of $\RA$ corresponding to the real hyperplane
section class $\Rh\in H^1(\RA)$. These orientations are well defined
over the whole group $\widetilde H_*(\RA)$ (Theorem 4.4.4).
\begin{remark}{{Remark} If $[\RA]$ is equal to the plane section class, manifold
$\RA$ may not possess any Stiefel orientation, but the modified
construction is still applicable to $\tR$. If $A$ is a surface, there
is another approach, also due to Viro. It provides a $\Pin^-$-structure
on $\RA$ which lifts to the 1-dimensional Stiefel orientation on $\tR$
arising due to Theorem 4.4.4. It would be interesting to find a
higher-dimensional generalization of this approach.
}\end{remark}
\begin{remark}{{Remark} The main difficulty in generalizing Viro's approach
(see the previous remark) seems to consist in the following:

In a way, $\Pin^-$-structure is the structure corresponding to 
characteristic class $w_2+w_1^2$. More precisely, let $f:X\to BO_n$
be a characteristic map of bundle $\xi$, and $P(\omega)\to BO_n$ be
a fixed fibration killing some characteristic class $\omega\in H^k(BO_n)$.
Then an $\omega$-structure on $\xi$ can be defined as a class
$\k\in H^{k-1}(f^*P(\omega))$ which does not vanish when restricted
to any fibre. (For example, Stiefel orientations are defined via
the Stiefel bundles $V_{n,k}(\eta_n)\to BO_n$, killing $w_{n-k+1}$; \
$\Pin^-$-structure is defined via the bundle $B\Pin^-\to BO_n$,
which kills $w_2+w_1^2$.) The problem is that there is no 
universal way to choose the killing fibration $P(\omega)$, so in fact
structures of this kind are functorially defined only modulo all
$(k-1)$-dimensional characteristic classes. This reduction being
accepted, both Viro's \ $\Pin^-$-structure and Nezvetaev's form (Viro \cite{8};
in fact, this form is a $w_1^2$-structure) can easily be generalized
(Degtyarev \cite{2}). But in order to define the absolute (i.e. not
reduced modulo characteristic classes) structures some appropriate
killing spaces over $BO_n$ and $BU_n$ are necessary which would be
functorial under some standard homomorphisms of groups $O_n$ and $U_n$.
}\end{remark}
\subsection*{ 0.3 }
Among the other results of the paper is a new approach to defining
Stiefel orientations. Instead of the associated Stiefel bundles
$V_{n,k}(\xi)$ it involves some other spaces which can be constructed
using only the total space of the associated sphere bundle $S^{n-1}(\xi)$
and the standard antipodal involution on it (Definition 3.3.1). It is this 
approach that, being applied to a tubular neighborhood of $F$ in $X$,
makes the construction of Theorem 4.1.1 possible.
\subsection*{ 0.4 }
The paper is organized as follows:

In section 1 we remind some known basic facts and introduce some notations,
which are not generally accepted.

Section 2 is devoted to $\Z$-spaces. Most of the notions and results of
this section seem to be familiar to those who deal with involutions,
though I could not find any systematic account of them. In 2.1 we discuss
singular Smith homology groups, which are more suitable for this paper.
In 2.2 the spectral sequence of space $S^\infty\tm_{\Z} X$ (cf., for example,
Hsiang \cite{4}) and of some natural finite dimensional approximations of
this space are introduced. In 2.3 we define the stabilized spectral 
sequence of an involution. This sequence, first introduced by
Kalinin \cite{5}, is, in fact, just a compact way to express the
stabilization theorem (Hsiang \cite{4}). Kalinin also conjectured
the description of the differentials of of these spectral sequences
given in 2.4. In 2.5 we define higher inclusion homomorphisms
$e_*^r:\SH_*(X)\dashrightarrow H_*(X)$ and study their properties.

In section 3 the new approach to constructing Stiefel orientations
is introduced and its relation to the ordinary orientations is studied.

Section 4 is devoted to the main construction and its applications to
real algebraic varieties.
\subsection*{Acknowledgements }
I take this opportunity to express my deep gratitude to O.Ya.Viro
for his help and interest to this work.

\section{Preliminaries }
\subsection{Complexes and homology groups }
For a topological space $X$ we denote by $S_*(X)$ and $S^*(X)$
its singular chain and cochain complexes with {\it coefficients
$\Z$}. All homology and cohomology groups also have 
coefficients $\Z$. When it cannot confuse, we use the same letters
to denote cycles and their homology classes.
\subsection{$G$-bundles }
Let $G$ be a Lie group. The term ``$G$-bundle'' (over space $X$)
means a principal $G$-bundle $\xi:G(\xi)\to X$, i.e. a free
right $G$-space $G(\xi)$ such that $G(\xi)/G=X$.

For a $G$-bundle $\xi$ and left $G$-space $F$ we denote by
$F(\xi)$ the {\it associated $F$-bundle\/} $G(\xi)\tm_GF\longto X$,
where the twisted product $G(\xi)\tm_GF$ is the quotient space
$$
G(\xi)\tm F\big/\left\{(x,f)=(xg^{-1},gf)\text{ for any }g\in G \right\}.
$$
\subsection{Some standard spaces }
We use the following standard notations:

$\R^n$ is the coordinate Euclidean space with the
standard basis $(\be_1,\dots,\be_n)$. For $k\le n$ we identify
$\R^k$ with the subspace of $\R^n$ generated by $(\be_1,\dots,\be_k)$;

$S^n$ and $D_{\pm}^n$ are the standard sphere and semispheres
\begin{align}
S^n&=\left\{\sum t^i\be_i\in\R^{n+1}\mid|\sum(t^i)^2=1\right\}\\
D_+^n&=\left\{\sum t^i\be_i\in\R^{n+1}\mid|\sum(t^i)^2=1,\;t^1\ge0\right\}\\
D_-^n&=\left\{\sum t^i\be_i\in\R^{n+1}\mid|\sum(t^i)^2=1,\;t^1\le0\right\};
\end{align}


$\Delta^n$ is the standard simplex
$\bigl\{\sum t^i\be_i\in\R^{n+1}\bigm|\sum t^i=1,\;t^i\ge0\bigr\}$;

$O_n$ is the group of orthogonal transformations of $\R^n$;

$V_{n,k}$ is the {\it Stiefel manifold,\/} which can be defined
as either the quotient space $O_n/O_{n-k}$, or the space of all
orthogonal $k$-frames in $\R^n$; it is well known that
$\widetilde H^i(V_{n,k})=0$ for $i<n-k$,
and $\widetilde H^{n-k}(V_{n,k})=\Z$
(see Steenrod, Epstein \cite{7});

for an $O_n$-bundle $\xi$ we denote by $S(\xi)$ the associated
$S^{n-1}$-bundle; $S(\xi)$ is called the {\it sphere bundle\/} of $\xi$.
Note that $S(\xi)=V_{n,1}(\xi)$.

Let $R=({\pmb x}_1,\dots,{\pmb x}_{k+1})$ be an orthonormal
$(k+1)$-frame in $\R^{n+1}$. Denote by $\Delta^k(R)$ the singular
simplex $\Delta^k\to S^n$, \
$\sum t^i{\pmb x}_i\longmapsto\sum t^i{\pmb x}_i\big/\sum(t^i)^2$.
We will use the following singular chains in $S^n$:
\begin{align}
\Sg_+^k&=\Delta^n(+\be_1,\dots,\be_{k+1})+
\sum_{i=2}^{k+1}\Delta^n(+\be_1,\dots,-\be_i,\dots,\be_{k+1}),\\
\Sg_-^k&=\Delta^n(-\be_1,\dots,\be_{k+1})+
\sum_{i=2}^{k+1}\Delta^n(-\be_1,\dots,-\be_i,\dots,\be_{k+1}),\\
\Sg^k&=\Sg_+^k+\Sg_-^k.
\end{align}
$\Sg_{\pm}^k$ will also be considered as singular chains in $D_{\pm}^k$.
If $c$ is the antipodal involution on $S^n$, then obviously
$c_\s\Sg_+^k=\Sg_-^k$, and $\d\Sg_{\pm}^k=\Sg_+^{k-1}+\Sg_-^{k-1}$.
\subsection{Stiefel orientations }
\begin{subsubsection}{ Definition} A ({\rm reduced\/}) $k$-{\rm dimensional
Stiefel orientation\/} on an $O_n$-bundle $\xi$ is a class
$\k_k\in H^k(V_{n,n-k}(\xi)$)
(resp., $\tilde\k_k\in\widetilde H^k(V_{n,n-k}(\xi))$),
such that for any fibre $V_{n,n-k}$ of fibration $V_{n,n-k}(\xi)\to X$
the restriction of $\k_k$ (resp., $\tilde\k_k$) to $\widetilde H^k(V_{n,n-k})$
is non-zero. The set of all (reduced) Stiefel orientations on $\xi$
is denoted by $\St_k(\xi)$ (resp., $\widetilde\St_k(\xi)$).

{\it Remark} A 0-dimensional Stiefel orientation is an ordinary
orientation of vector bundle $\xi$. A 1-dimensional orientation is
a $\Pin^+$-structure on $\xi$. A pair consisting of a 0- and
1-dimensional orientation is a $\Spin$-structure.

{\it Remark} If $k>0$, obviously $\widetilde\St_k(\xi)=\St_k(\xi)$.
If $k=0$, a reduced orientation is a pair of opposite ordinary
orientations on $\xi$. (So this notion is not trivial only if
the underlying space $X$ is not connected.)
}\end{remark}
\subsubsection{Theorem \textrm{(well known)}} {\it Let $\xi$ be an
$O_n$-bundle over $X$. Then $\St_k(\xi)$ and $\widetilde\St_k(\xi)$
are not empty if and only if $w_{k+1}(\xi)=0$. In this case
$\St_k(\xi)$ (resp., $\widetilde\St_k(\xi)$) is an affine space
over $H^k(X)$ (resp., $\widetilde H^k(X)$). The reduction map
$\St_k(\xi)\to\widetilde\St_k(\xi)$ is affine.
}
%
%     The end of file 1.
%     To be followed by file 2.
%     File 2 of
% ``Stiefel orientations on a real algebraic variety''
%    by A.I.Degtyarev
%
\section{ $\Z$-spaces }
\subsection{Singular Smith homology }
Throughout this section $X$ is a fixed {\it $\Z$-space,} i.e.
a topological space supplied with an involution $c:X\to X$.
We denote by $\Fix c$ the fixed point set of $c$, and by $X/c$
the quotient space. For simplicity we assume that pair $(X,\Fix c)$
allows a $c$-invariant cell partition. (Note, though, that most
results can easily be extended to general $\Z$-spaces using an
equivariant cell approximation of $(X,c)$.)
\subsubsection{Notation} Denote $\L=\Z[c]/(c^2=1)$.
Since $\Z$ acts on $S_*(X)$ via the induced involution $c_\s$,
this complex can be considered as a complex of $\L$-modules.

We will also consider the $\Z$-algebra $H^*(\Rp^k)=\Z[h]/(h^{k+1}=0)$
and the $H^*(\Rp^k)$-module $H_*(\Rp^k)$. The generator of
$H_i(\Rp^k)=\Z$ is denoted by $h_i$; both $h_i$ and $h^i$ are
assumed to be zero if either $i<0$ or $i>k$.

\subsubsection{Definition} The {\it Smith homology groups\/}
$\SH_*(X)$ are the homology groups of the {\it Smith complex\/}
$\Sm_*(X)=\Ker[(1+c_\s):S_*(X)\to S_*(X)]$. The inclusion
homomorphism $\SH_*(X)\to H_*(X)$ is denoted by $\sm_X$.

\subsubsection{Proposition} {\it (1) Let $C_*(X)$ be the cell complex
corresponding to some $c$-invariant cell partition of pair $(X,\Fix c)$.
Then $\SH_*(X)=H_*(\Ker[(1+c_\s):C_*(X)\to C_*(X)])$;

(2) denote by $p:X\to X/c$ the projection. Then the map
$x\mapsto \rel p_\s x$ induces isomorphism
$H_*(S_*(X)/\Sm_*(X))\cong H_*(X/c,\Fix c)$;

(3) $\Sm_*(X)$ naturally splits into
$S_*(\Fix c)\oplus S_*(X)/\Sm_*(X)$, due to (2) this induces an
isomorphism $\SH_*(X)\cong H_*(\Fix c)\oplus H_*(X/c,\Fix c)$.
}

\begin{proof}{ The last two statements are well known for cell
Smith homology groups (cf. Bredon \cite{1}), so they follow
from the first one. Proof of the first statement is absolutely
analogous to the standard proof of similar statement about
ordinary homology groups $H_*(X)$. We omit the details since
they would require developing from the very beginning the theory
of functor $\SH_*$, which is absolutely analogous to the standard
singular homology theory. 
}\end{proof}
\subsubsection{Corollary} {\it Let $(X,\Fix c)$ is an $n$-dimensional
cell pair. Then $\SH_i(X)=0$ for $i>n$, and
$\SH_n(X)=\Ker[(1+c_*):H_n(X)\to H_n(X)]$.
}

\subsubsection{Corollary} {\it Let $X$ be an $n$-dimensional smooth
manifold, and $c$ be a smooth involution. Then there exists one
and only one class $[X]_S\in\SH_n(X)$ such that
$\sm_X[X]_S=[X]$. We call this class the {\rm Smith fundamental class}
of $X$.)
}

For $(\sm_X)_n:\SH_n(X)\to H_n(X)$ is a monomorphism, and
$[X]$ is a $c_*$-invariant class. 
\subsection{ Spectral sequence of an involution }
\begin{definition}{{2.1.1. Definition} Denote by $\CS^kX$, \ $0\le k\le\infty$,
the twisted product $S^k\tm_{\Z}X$ supplied with the induced involution
and filtration by subspaces
$\emptyset\subset \CS^0X\subset \CS^1X\subset\dots\subset \CS^kX$.
(Here inclusions are induced by the standard inclusions
$\emptyset\subset S^0\subset S^1\subset\dots\subset S^k$.)
The homology and cohomology spectral sequences of filtered
space $\CS^kX$ are denoted by $\E k/r/pq/(X)$ and $\E k/pq/r/(X)$
respectively and called the {\it spectral sequences\/} of
involution $c$. (Here and in other similar notations space
$X$ will often be omitted.)
}\end{definition}
Note that if $X$ is a trivial $\Z$-space, i.e. $c=\id_X$,
then $\CS^kX=\Rp^k\tm X$, and
$H_*(\CS^kX)=H_*(\Rp^k)\o H_*(X)$. In particular,
both the spectral sequences degenerate in term $E^1$.
\begin{proclaim}{{2.2.2. Proposition}

(1) $\E k/1/**/=H_*(\Rp^k)\o H_*(X)$, and
$\E k/**/1/=H^*(\Rp^k)\o H^*(X)$;

(2) $d^1(h_p\o x)=h_{p-1}\o(1+c_*)x$, and $d_1(h^p\o x)=h^{p+1}\o(1+c^*)x$;

(3) if $r\ge 2$, both the sequences coincide with the Serre
spectral sequences of fibration $\CS^kX\to \Rp^k=\CS^k\pt$;

(4) $\E k/**/r/$ is a bigraded $H^*(\Rp^k)$-algebra,
and $\E k/r/**/$ is a bigraded $\E k/**/r/$-module;
if $r\ge2$, differentials $d^r$ and $d_r$ are derivatives;

(5) $\E k/r/pq/$ and $\E k/pq/r/$ converge to 
$H_{p+q}(\CS^kX)$ and $H^{p+q}(\CS^kX)$ respectively;
the convergence observes both the multiplicative structures;

(6) if $l\le k$, the inclusion homomorphism
$\E l/r/pq/\to \E k/r/pq/$ (resp., $\E k/pq/r/\to \E l/pq/r/$)
is surjective (resp., injective) for $p\le l$ and
one-to-one for $p\le l+1-r$.
}\end{proclaim}
\begin{proof}{ By definition,
\begin{align}
\E k/1/pq/&=H_{p+q}(\CS^pX,\CS^{p-1}X)\\
&=H_{p+q}(S^p\tm X,D_-^p\tm X)\\
&=H_{p+q}(D_+^p\tm X,S^{p-1}\tm X)\\
&=H_q(X).
\end{align}
This proves (1). Statement (2) follows from 2.4.4(1) below.
(3) is just a rewording of the construction of the Serre spectral sequence
using the skeleton filtration of the base of the fibration.
(4) and (5) are well known properties of the Serre spectral
sequence. (6) follows from the fact that $\E l/1/**/$
(resp., $\E l/**/1/$) is a truncation of $\E k/1/**/$
(resp., $\E k/**/1/$). 
}\end{proof}
\subsection{Stabilized spectral sequence }
Denote by $\f_p^r:\E\infty/r/p+1,*/\to\E\infty/r/p,*/$ and
$\f_r^p:\E\infty/p,*/r/\to\E\infty/p+1,*/r/$ the maps
$x\mapsto h\cap x$ and $x\mapsto h\cup x$ respectively.
\begin{definition}{{2.3.1. Definition \rm(Kalinin \cite{5})}
Spectral sequences
$$\E/r/q/(X)=\varprojlim(\E\infty/r/pq/,\f_p^r)\quad
\text{and}\quad\E/q/r/(X)=\varinjlim(\E\infty/pq/r/,\f_r^p)
$$
are called the {\it stabilized spectral sequences\/} of involution $c$.
}\end{definition}

Natural maps $\E/r/q/\to\E\infty/r/pq/$ and $\E\infty/pq/r/\to\E/q/r/$
will be denoted by $\psi_q^r$ and $\psi_r^q$ respectively.
\begin{proclaim}{{2.3.2. Theorem \rm (Kalinin \cite{5})}

(1) $E_q^1=H_q(X)$, and $E_1^q=H^q(X)$;

(2) $d^1=1+c_*$, and $d_1=1+c^*$;

(3) $E_r^*$ is a graded $\Z$-algebra, and $E_*^r$ is a graded
$E_r^*$-module; $d_r$ and $d^r$ are derivatives if $r\ge2$;

(4) provided that $X$ is a finite $CW$-complex,
$E_*^r$ and $E_r^*$ converge to $H_*(\Fix c)$ and $H^*(\Fix c)$
respectively; the convergence observes the multiplicative structures,
but it does not observe the grading of $H_*(\Fix c)$ and $H^*(\Fix c)$;

(5) homology maps $\psi_p^r$ are injective;
cohomology maps $\psi_r^p$ are surjective.
}\end{proclaim}
\begin{definition}{{2.3.3. Notation} The filtration on $H_*(\Fix c)$ which
arises due to convergence $E_*^r\Rightarrow H_*(\Fix c)$ is denoted by
$\F_*(X)$. The composed maps
$\F_q\longto\F_q/\F_{q+1}@>\cong>> E_q^\infty$ are denoted by $e_q$.

Note that filtration $\F_q$ can as well be defined if $E_*^r$ does
not converge to $H_*(\Fix c)$ (Kalinin \cite{5}). By definition,
the class of cycle $x=\sum x_i$, \ $x_i\in S_i(\Fix c)$, belongs to
$\F_q$ if for some $p$ the image of
$\sum h_{p+q-i}\o x_i\in H_{p+q}(\CS^\infty\Fix c)$  in
$H_{p+q}(\CS^\infty X)$ comes from $H_{p+q}(\CS^pX)$.
}\end{definition}
\subsection{Geometric description }
Let $C_*$ (resp., $C^*$) be a chain (cochain) complex of
$\L$-modules. Define filtered complexes $\CS^kC_*$ and $\CS^kC^*$
as follows:

as a graduated group $\CS^kC_*=H_*(\Rp^k)\o C_*$
(resp., $\CS^kC^*=H^*(\Rp^k)\o C^*$);

the boundary (coboundary) operator is the map
$h_i\o x\mapsto h_i\o\d x+h_{i-1}\o(1+c)x$
(resp.,\linebreak $h^i\o x\mapsto h^i\o\cd x+h^{i+1}\o(1+c)x$);

the filtration is, respectively,
$F_p=\Im[\inc_*\o\id:\CS^pC_*\to \CS^kC_*]$, and
$F^p=\Ker[\inc^*\o\id:\CS^kC^*\to \CS^pC^*]$.

The spectral sequences of filtered complexes $\CS^kC_*$ and $\CS^kC^*$ 
are denoted by $\E k/r/pq/(C_*)$ and $\E k/pq/r/(C^*)$ respectively.
\begin{proclaim}{{2.4.1. Theorem} There exists a natural chain map
$\t:\CS^kS_*(X)\to S_*(\CS^kX)$ which induces isomorphisms of
\begin{itemize}
\item spectral sequences $\E k/r/**/(S_*(X))\cong \E k/r/**/(X)$
and $\E k/**/r/(S^*(X))\cong \E k/**/r/(X)$;
\item homology groups $H_*(\CS^kS_*(X))\cong H_*(\CS^kX)$ and
$H^*(\CS^kS^*(X))\cong H^*(\CS^kX)$;
\item Smith homology groups $H_*(\Rp^k)\o\SH_*(X)\cong\SH_*(\CS^kX)$.
\end{itemize}
}\end{proclaim}
\begin{proof}{ Let $p$ be the projection $S^k\tm X\to \CS^kX$, and
$\tilde c$ be the diagonal involution on $S^k\tm X$. Fix some natural
chain homotopy equivalence $\mu:S_*(X)\o S_*(Y)\longto S_*(X\tm Y)$
and define $\t$ as the map $h_i\o x\mapsto p_\s\mu(\Sg_+^i\o x)$.
Since $p_\s\tilde c_\s=p_\s$, and $\mu$ is natural,
\begin{align}
&\t\d(h_i\o x)-\d\t(h_i\o x)=\\
&\qquad=p_\s\mu(\Sg_+^{i-1}\o c_\s x+\Sg_+^i\o\d x)-
  p_\s\mu(\Sg_+^{i-1}\o x+\Sg_-^{i-1}\o x+\Sg_+^i\o\d x)=\\
&\qquad=p_\s\mu(\Sg_+^{i-1}\o c_\s x-\Sg_-^{i-1}\o x)=
p_\s(1-\tilde c_\s)\mu(\Sg_+^{i-1}\o c_\s x)=0.
\end{align}
This proves that $\t$ is a chain map. The subquotient map
$$
\t_p:\CS^pS_*(X)/\CS^{p-1}S_*(X)=h_p\o S_*(X)\longto S_*(\CS^pX,\CS^{p-1}X)
$$
can be decomposed through the map
$$
\tilde\t_p:h_p\o S_*(X)\longto S_*(D_+^p\tm X,S^{p-1}\tm X),\\
h_p\o x\mapsto\mu(\Sg_+^i\o x),
$$
which induces an isomorphism of terms $E^1$ of the corresponding spectral
sequences. Hence $\t$ induces isomorphisms of the spectral sequences
and homology groups.

To prove the last statement, we need the following lemma:
\begin{proclaim}{{2.4.2. Lemma} Let $f:C_*\to D_*$ be a chain map
which induces isomorphism $f_*:H_*(C_*)\to H_*(D_*)$. Then
$(\id\o f)_*:H_*(\CS^kC_*)\to H_*(\CS^kD_*)$ is also an
isomorphism for any $k$.
}\end{proclaim}
\begin{proof} $\id\o f$ induces an isomorphism of terms $E^1$
of spectral sequences $\E k/r/**/(C_*)$ and $\E k/r/**/(D_*)$. 
\end{proof}
Lemma 2.4.2 and Proposition 2.1.3(2) imply that $\t$ induces
isomorphism
$$
H_*(\CS^k(S_*(X)/\Sm_*(X)))\cong H_*(S_*(\CS^kX)/\Sm_*(\CS^kX)),
$$
and, due to Five Lemma, isomorphism
$$
H_*(\CS^k\Sm_*(X))\cong\SH_*(\CS^kX).
$$
Since $c_\s$ acts trivially on $\Sm_*(X)$, the first group is equal to
$H_*(\Rp^k)\o\SH_*(X)$. \qed
\begin{proclaim}{{2.4.3. Corollary \rm (of 2.4.1 and 2.4.2)} Theorem 2.4.1
remains valid if $S_*(X)$ is replaced with the cell complex
corresponding to some $c$-invariant cell partition of pair
$(X,\Fix c)$. (Note that for (1) and (2) it even is not necessary
that $\Fix c$ be a cell subcomplex of $X$.)
}\end{proclaim}
The remaining statements provide a clear geometric description of
spectral sequences $\E k/r/**/$ and $E_*^r$. (The dual description
of cohomology spectral sequences is also valid.) The term ``chain''
(``cycle'') means either singular or cell chain (resp., cycle).
\begin{proclaim}{{2.4.4. Corollary} Let $x$ be a $q$-dimensional cycle in $X$. Then

(1) $d^1(h_p\o x)=h_{p-1}\o(1+c_\s)x$;

(2) let $r\le p$. Then $d^r(h_p\o x)$ vanishes in $\E k/r/p-r,q+r-1/(X)$
if and only if there exist some $i$-dimensional chains $y_i$ in $X$, \
$q+1\le i\le q+r$, such that $(1+c_\s)x=\d y_{q+1}$,
and $(1+c_\s)y_i=\d y_{i+1}$ for $i\le q+r-1$. In this case
$d^{r+1}(h_p\o x)=h_{p-r-1}\o(1+c_\s)y_{q+r}$.
}\end{proclaim}
\begin{proclaim}{{2.4.5. Corollary} Let $x$ be a $q$-dimensional cycle in $X$. Then

(1) $d^1x=(1+c_\s)x$;

(2) $d^rx$ vanishes in $\E/r/q+r-1/(X)$ if and only if there exist some
$i$-dimensional chains $y_i$ in $X$, \ $q+1\le i\le q+r$, such that
$(1+c_\s)x=\d y_{q+1}$, and $(1+c_\s)y_i=\d y_{i+1}$ for $i\le q+r-1$.
In this case $d^{r+1}x=(1+c_\s)y_{q+r}$.
}\end{proclaim}
\begin{proof}{ 2.4.4 is, in fact, the definition of the spectral
sequence of complex $\CS^kS_*(X)$. 2.4.5 is obtained
by stabilizing differentials of $\E\infty/r/**/$. 
}\end{proof}
\begin{proclaim}{{2.4.6. Proposition} Let $\i:\Fix c\into X$ be the inclusion.
Consider some $i$-dimensional cycles $x_i$  in $\Fix c$, and denote
$x=\sum x_i$. Then $x\in\F_q(X)$ if and only if there exist some
$i$-dimensional chains $y_i$ in $X$, \ $i\le q$, such that
$\i_\s x_i=\d y_{i+1}+(1+c_\s)y_i$ for $i\le q+1$. In this case
$e_qx=\i_\s x_q+(1+c_\s)y_q$.
}\end{proclaim}
\begin{proof}{ Replace $S_*(\CS^kX)$ with $\CS^kS_*(X)$. By definition
(cf. 2.3.3) $x\in\F_q$ if and only if 
$\i_\s(\sum h_{p+q-i}\o x_i)=\bar x+\d y$ for some $p\ge0$, \
$\bar x\in \CS^pS_{p+q}(X)$, and $y\in \CS^\infty S_{p+q+1}(X)$.
In this case $\psi_pe_qx$ is the class of $\bar x$ in $\E\infty/\infty/pq/$. 
Expanding $y$ in the sum $\sum h_{p+q+1-i}\o y_i$ we obtain that
$\i_\s x_i=\d y_{i+1}+(1+c_\s)y_i$ for $i\le q+1$, and
\begin{align}
\bar x&=h_p\o(\i_\s x_q+(1+c_\s))y_q\\
      &+\d(\sum_{i\ge q+1}h_{p+q+1-i}\o y_i)
&\qquad\text{(belongs to }\d \CS^pS_*(X))\\
      &+\sum_{i\ge q+1}h_{p+q-i}\o\i_\s x_i
&\qquad\text{(belongs to }\CS^{p-1}S_*(X)).
\end{align}

Hence $\psi_pe_qx=h_p\o(\i_\s x+(1+c_\s)y_q)$. 
}\end{proof}
\subsection{Higher inclusion maps }
Define filtration $\SF_q^r(X)$ on $\SH_*(X)$ and homomorphisms 
$e_q^r:\SF_q^r(X)\to E_q^r(X)$ as follows:

The class of a cycle $x=\sum\limits_{i\ge q-r+1}x_i$, \ $x_i\in\Sm_i(X)$,
belongs to $\SF_q^r$ if and only if there exist some chains $y_i\in S_i(X)$, \
$q-r+2\le i\le q$, such that $x_{q-r+1}=\d y_{q-r+2}$, and
$x_i=\d y_{i+1}+(1+c_\s)y_i$ for $q-r+1\le i\le q-1$. In this case $e_q^rx$
is the class of $x_q+(1+c_\s)y_q$ in $E_q^r(X)$.

Denote by $\F_q^r(X)$ the filtration $\SF_q^r\cap H_*(\Fix c)$
on $H_*(\Fix c)$ (cf. 2.1.3(3)). Restriction
$e_q^r\left|{\vphantom{(_(}}\right._{\F_q^r}$ will also be denoted by $e_q^r$.
\begin{definition}{{2.5.1. Definition} Homomorphisms $e_q^r$ are called
the {\it higher inclusion homomorphisms.}
}\end{definition}
\begin{remark}{{Remark} In this definition singular chains can as well be replaced
with cell chains corresponding to some $c$-invariant cell partition of
pair $(X,\Fix c)$.
}\end{remark}
\begin{proclaim}{{2.5.3. Proposition} Homomorphisms $e_q^r$ are well defined.
}\end{proclaim}
\begin{proof}{ We have to prove the following three statements:
(1) $x_q+(1+c_\s)y_q$ is a cycle; (2) $d^s(x_q+(1+c_\s)y_q)=0$ for $s\le r$,
so this cycle projects into $E_q^r$; (3) the class of $x_q+(1+c_\s)y_q$
in $E_q^r$ does not depend on the choice of $y_i$'s.
\begin{enumerate}\item{(3)}
\item"(1)" $\d(x_q+(1+c_\s)y_q)=
(1+c_\s)x_{q-1}+(1+c_\s)^2y_{q-1}=0$
since $x_{q-1}$ is $c_\s$-invariant, and $(1+c_\s)^2=0$;
\item"(2)" $d^s(x_q+(1+c_\s)y_q)=0$ for any $s$ due to the fact that
this cycle is $c_\s$-invariant (see Corollary 2.4.5);
\item"(3)" let $y_i'=y_i+z_i$ be some other chains. Then $\d z_{q-r+2}=0$,
and $(1+c_\s)z_i=\d z_{i+1}$ for $q-r+2\le i\le q-1$. Hence
$(1+c_\s)y_i'=(1+c_\s)y_i+d^{r-1}z_{q-r+2}$ (Corollary 2.4.5), and
the image of $x_q+(1+c_\s)y_i'$ in $E_q^r$ coincides with that of
$x_q+(1+c_\s)y_i$. 
\end{enumerate}
}\end{proof}
\begin{proclaim}{{2.5.4. Theorem}
\begin{itemize}
\item $\SF_q^1=\bigoplus\limits_{i\ge q}\SH_i(X)$, and
$\SF_q^r=\Ker e_{q-1}^{r-1}$;
\item $\SF_q^r\supset\bigoplus\limits_{i\ge q}\SH_i(X)$;
restriction of $e_q^r$ to $\SH_i(X)$ is trivial for
$i>q$ and coincides (in the obvious sense) with $\sm_X$ for $i=q$;
\item $\SF_q^r\subset\SF_q^{r+1}$; restriction of $e_q^{r+1}$ to
$\SF_q^r$ coincides with $e_q^r$;
\item statements (1)--(3) remain valid if $\SF_q^r$ are
replaced with $\F_q^r$;
\item $\F_q^r=\F_q$ and $e_q^r=e_q$ for any $r\ge q+2$
(note that in this case $E_q^r=E_q^\infty$.
\end{itemize}
}\end{proclaim}
\begin{proof}{ If $x\in\SF_q^r$, then obviously $x\in\SF_{q-1}^{r-1}$,
and $e_{q-1}^{r-1}x=x_{q-1}+(1+c_\s)y_{q-1}=\d y_q$ vanishes in
$E_{q-1}^{r-1}$. Hence $\SF_q^r\subset\Ker e_{q-1}^{r-1}$.

Let $x\in\Ker e_{q-1}^{r-1}$. This means that 
$x_{q-1}+(1+c_\s)y_{q-1}=d^{r-2}z_{q-r+2}+\d y_q'$ for some cycle
$z_{q-r+2}\in S_{q-r+2}(X)$ and chain $y_q'\in S_q(X)$. Due to Corollary 2.4.5
this is equivalent to existence of some chains $z_i$, \ $q-r+3\le i\le q-1$,
such that $\d z_i=(1+c_\s)z_{i-1}$, and
$x_{q-1}+(1+c_\s)y_{q-1}=(1+c_\s)z_{q-1}+\d y_q'$.
Denoting $y_i'=y_i+z_i$ for $q-r+2\le i\le q-1$,
we see that $x\in\SF_q^r$. This completes proof of (1).

Statements (2)--(4) immediately follow from the definitions,
and (5) follows from comparing the definition of
$\SF_q^r$ and Proposition 2.4.6. 
}\end{proof}
\begin{definition}{{2.5.5. Definition} Define map
$\bar e_q^r:\SF_q^r\to H_q(\CS^{r-1}X)$ as the product
$$\begin{CD}
\SF_q^r@>e_q^r>>E_q^r@>\psi_0^r>>\E\infty/r/0,q/
@<\cong<<\E r-1/r/0,q/@>>>H_q(\CS^{r-1}X)\end{CD}
$$
(here the third map is invertible due to 2.2.2(6); the last map is the edge
homomorphism of spectral sequence $\E r-1/r/pq/$).
}\end{definition}
\begin{proclaim}{{2.5.6. Proposition} Let $x=\sum\limits_{i\ge q}x_i$, \
$x_i\in\SH_i(X)$, and $r\ge0$. Then
$\sm_{\CS^rX}(\sum h_{q+r-i}\o x_i)=0$ in $H_{q+r}(\CS^rX)$
if and only if $x\in\SF_{q+r+1}^{r+2}$. In this case
$\sm_{\CS^{r+1}X}(\sum h_{q+r+1-i}\o x_i)=\bar e_{q+r+1}^{r+2}(x)$.
}\end{proclaim}
\begin{proof}{ According to the definition the condition
$x\in\SF_{q+r+1}^{r+2}$ is equivalent to the equality
$\sum h_{q+r-i}\o x_i=\d(\sum h_{q+r+1-i}\o y_i)$ for some
$y_i\in S_i(X)$ (cf. Corollary 2.4.4). In this case
$\sum h_{q+r+1-i}\o x_i=\d(\sum h_{q+r+2-i}\o y_i)+
h_0\o(x_{q+r+1}+(1+c_\s)y_{q+r+1})$; hence this element
projects to $\bar e_{q+r+1}^{r+2}x\in H_{q+r+1}(\CS^{r+1}X)$. 
}\end{proof}
\begin{remark}{{Remark} Since $\psi_0^r$ and the edge homomorphisms
$\E r-1/r/pq/\to H_q(\CS^{r-1}X)$ are injective, Proposition 2.5.6 can
as well be taken for definition of $\SF_q^r$ and $e_q^r$.
}\end{remark}
%
%   The end of file 2.
%   To be followed by file 3.

%   File 3 of
% ``Stiefel orientations on a real algebraic variety''
%    by A.I.Degtyarev
%
\section{The basic construction }
\subsection{Groups $H^{n-1}(\CS^kS(\xi))$}
Throughout this section $\xi$ denotes some fixed
$O_n$-bundle over space $X$.
\begin{definition}{{3.1.1 Definition} Denote by $\eta_k$ the tautological linear
bundle over $\Rp^{k}$, and by $\eta_k\o\xi$ the exterior tensor
product, which is a bundle over $\Rp^{k}\tm X$.
}\end{definition}
\begin{proclaim}{{3.1.2. Proposition} (1) There exists a natural fibrewise
homeomorphism $S(\eta_k\o\xi)=\CS^kS(\xi)$, \ $S(\xi)$ being considered as
a $\Z$-space via the standard antipodal involution;

(2) if space $X$ is connected and $k\le n-1$,
there exists a natural exact sequence
$$
0\to\sum\limits_{i=n-k-1}^{n-1}H^i(X)\to H^{n-1}(\CS^kS(\xi))
\overset{i^*}\to{\longto}H^{n-1}(S^{n-1}),
$$
where $i^{*}$ is induced by the inclusion of a fibre $S^{n-1}$ of
fibration $S(\eta_k\o\xi)\to\Rp^{k}\tm X$;

(3) homomorphism $i^{*}$ is non-zero if and only if
$w_n(\xi)=\dots=w_{n-k}(\xi)=0$;

(4) if $l\le k$, the following diagram commutes
$$\require{AMScd}
\begin{CD}
0@>>>\bigoplus\limits_{i=n-k-1}^{n-1}H^i(X)@>>>H^{n-1}(\CS^kS(\xi))@>i^*>>
   H^{n-1}(S^{n-1})\\
@.@VV\text{projection}V@VV\text{restriction}V@VV\text{identity}V\\
0@>>>\bigoplus\limits_{i=n-l-1}^{n-1}H^i(X)@>>>H^{n-1}(\CS^lS(\xi))@>i^*>>
   H^{n-1}(S^{n-1}),
\end{CD}
$$
}\end{proclaim}
\begin{proof}{
Statement (1) immediately follows from definitions. To prove (2) and
(3) consider the Gisin exact sequence of the sphere bundle
$S(\eta_k\o\xi)\to\Rp^{k}\tm X$:
$$\require{AMScd}
\minCDarrowwidth{1.5em}
\begin{CD}
H^{n-1}(\Rp^k\tm X)&\hookrightarrow&H^{n-1}(S(\eta_k\o\xi))
   @>>>H^0(\Rp^k\tm X)@>e\cup>>H^n(\Rp^k\tm X)\\
@|&\searrow^{i^*}&@VV\cong V\\
\bigoplus\limits_{i=n-k-1}^{n-1}H^i(X)@.@.H^{n-1}(S^{n-1})
\end{CD}
$$
where $e=\sum h^{n-i}\o w_i(\xi)$ is the characteristic class of the bundle.
It yields that $\Im i^{*}\ne 0$ if and only if $e=0$.
Statement (4) follows from the naturallity of the Gisin sequence. 
}\end{proof}
\subsection{Maps $\chi^{k}:H^{n-1}(\CS^{k-1}S(\xi))\longto
\bigoplus\limits_{i=1}^k H^{n-i}(V_{n,i}(\xi))$ }
\begin{definition}{{3.2.1. Definition} Let $f:Y\to V_{n,k}(\xi)$ be some map,
$y\mapsto(x(y),\be_1(y),\dots,\be_k(y))$, where $x(y)\in X$ and
$(\be_1(y),\dots,\be_k(y))$ is a $k$-frame in the fibre at $x(y)$ of the
associated vector bundle $\R^{n}(\xi)$. Define the {\it coordinatewise lift\/}
of $f$ as the map $\tilde f:Y\tm S^{k-1}\to S(\xi)$, \
$y\tm\sum t^{i}\be_i\longmapsto(x(y),\sum t^{i}\be_i(y))$.
(Here $S^{k-1}$ and $S(\xi)$ are considered as subspaces of
$\R^{k}$ and $\R^{n}(\xi)$ respectively.)
}\end{definition}
\begin{definition}{{3.2.2. Definition}
Fix some natural chain homotopy equivalence
$$\mu: S_*(X)\o S_*(Y)\longto S_*(X\tm Y)$$
and define the following maps:

(1) $\t_{ij}:S_i(V_{n,k}(\xi))\longto S_{i+j}(S(\xi))$, \ $j\le k-1$.
Let $x=\sum\sg_{\alpha}$ be a singular chain, i.e. a sum of singular simplexes
$\sg_{\alpha}:\Delta^{i}\to V_{n,k}(\xi)$. Then $\t_{ij}(x)=0$ for $j<0$, and
$\t_{ij}(x)=\sum(\tilde\sg_{\alpha})_{\s}\mu(\id_{\Delta^{i}}\o\Sg_{+}^{j})$
otherwise;

(2) $\bar\f_{k,i}:S_i(V_{n,k}(\xi))\longto \CS^{k-1}S_{i+k-1}(S(\xi))$.
By definition, $\bar\f_{k,i}(x)=\sum h_{k-j-1}\o\t_{ij}(x)$.
}\end{definition}
\begin{proclaim}{ {3.2.3. Proposition} (1) $\bar\f_{k,i}$ is a chain map, so it
induces homomorphisms
$$
\f_k=(\bar\f_{k,n-k})_*:H_{n-k}(V_{n,k}(\xi)))\longto
   H_{n-1}(\CS^{k-1}S(\xi)),\qquad\text{and}\\
\f^{k}:H^{n-1}(\CS^{k-1}S(\xi))\longto H^{n-k}(V_{n,k}(\xi));
$$

(2) if $X=\pt$, homomorphism $\f^{k}:H^{n-1}(\CS^{k-1}S^{n-1})\longto
H^{n-k}(V_{n,k})$ is one-to-one;

(3) the following diagram commutes (cf. 3.1.2(2))
$$\begin{CD}
H^{n-k}(X)@>>>H^{n-1}(\CS^{k-1}S(\xi))\\
&\searrow&@VV\f^kV\\
@.H^{n-k}(V_{n,k}(\xi)).
\end{CD}
$$
}\end{proclaim}
\begin{proof}{
We prove the dual statements about homology maps
$\f_k$. Denote by $c$ the antipodal involution on $S(\xi)$.

Let $f:Y\to V_{n,k}(\xi)$ be some map, and $x\in S_i(Y)$ be a singular chain
in $Y$. Then obviously $\t_{ij}(f_\s x)=\tilde f_\s\mu(x\o\Sg_+^{j})$.
In particular, for a singular simplex $\sg:\Delta^{i}\to V_{n,k}(\xi)$
this implies that $\d\t_{ij}(\sg)=\tilde\sg_\s\mu(\d\id_{\Delta^{i}}
\o\Sg_+^{j}+\id_{\Delta^{i}}\o(\Sg_+^{j-1}+\Sg_-^{j-1}))=
\t_{i-1,j}(\d\sg)+(1+c_\s)\t_{ij}(\sg)$,
and $\d\bar\f_{k,i}(\sg)=\bar\f_{k,i}(\d\sg)$.
This proves statement (1).

To prove (2), consider the inclusion $\iota:S^{n-k}\to V_{n,k}$:
$$
\sum t^{i}\be_i\longmapsto(\be_1,\dots,\be_{k-1},\sum t^{i}\be_{i+k-1}),
$$
which induces isomorphism $\iota_*:H_{n-k}(S^{n-k})\to H_{n-k}(V_{n,k})$
(cf. Steenrod, Epstein \cite{7}); we suppose that $k<n$). It is easy to see
that the coordinatewise lift $\tilde\iota:S^{n-k}\tm S^{k-1}\to S^{n-1}$
is a degree 1 map, whose restriction to $S^{n-k}\tm S^{k-2}$ coincides with the
product $\const\tm\inc:S^{n-k}\tm S^{k-2}\longto\pt\tm S^{n-1}$.
Hence $\t_{n-k,j}(\Sg^{n-k})=\mu(\const_\s\Sg^{n-k}\o
\inc_\s\Sg_+^{j})=0$ for $j\le k-2$, and $\t_{n-k,k-1}(\Sg^{n-k})$
is a fundamental cycle of $S^{n-1}$, which generates $H_{n-1}(S^{n-1})$
(cf.3.1.2). If $k=n$, the same construction applied separately
to each cycle $\Sg_{\pm}^{0}$ in $S^{0}$ gives both the generators of
$H_{n-1}(\CS^{n-1}S^{n-1})=\Z\oplus\Z$.

Prove statement (3). Let $\sg:\Delta^{i}\to V_{n,k}(\xi)$ be a singular
simplex. Denote by $p$ the projection $V_{n,k}(\xi)\to X$ and by $\sg_X$
the simplex $p\circ\sg:\Delta^{i}\to X$, and consider the following
commutative diagram
$$\begin{CD}
\Delta^i\tm S^{n-1}@>{\tilde\sg}>>S(\xi)\\
@V\id\tm\const VV@VVp_S V\\
\Delta^i\tm\pt@>\sg_X\tm\id>>X\tm\pt
\end{CD}
$$
Since $\mu$ is natural, $(p_S)_\s\t_{ij}(\sg)=\mu(\sg_X\o\const_\s\Sg_+^j)$.
If $j>0$, \ $\const_\s\Sg_+^j=0$. Hence $\bar\f_{k,n-k}$ projects to the map
$\sg\mapsto h_{k-1}\o\mu(p_\s\sg\o\id_{\pt})$, which induces isomorphism
$H_{n-k}(X)\longto h_{k-1}\o H_{n-k}(X)$.
\begin{definition}{{3.2.4. Definition} Define $\chi_k:H^{n-1}(\CS^{k-1}S(\xi))
\longto\bigoplus\limits_{i=1}^k H^{n-i}(V_{n,i}(\xi))$
as the direct sum of composed maps
$$\begin{CD}
H^{n-1}(\CS^{k-1}S(\xi))@>\text{restriction}>>
  H^{n-1}(\CS^{i-1}S(\xi))@>\f^i>>H^{n-i}(V_{n,i}(\xi)).
\end{CD}
$$
}\end{definition}
\begin{proclaim}{{3.2.5. Proposition} For any $l\le k\le n$
the following diagram commutes
$$\begin{CD}
\bigoplus\limits_{i=1}^kH^{n-1}(X)@>>>H^{n-1}(\CS^{k-1}S(\xi))
   @>\inc^*>>H^{n-1}(\CS^{l-1}S(\xi))\\
&\searrow&@VV{\chi^k}V@VV{\chi^l}V\\
@.\bigoplus\limits_{i=1}^kH^{n-i}(V_{n,i}(\xi))
   @>\text{projection}>>\bigoplus\limits_{i=1}^lH^{n-i}(V_{n,i}(\xi))
\end{CD}
$$
}\end{proclaim}
\begin{proof} Immediately follows from 3.2.3(3). 
\end{proof}
\subsection{Flag structures }
\begin{definition}{{3.3.1. Definition} A $k$-{\it flag structure\/} on bundle $\xi$
is a class $\k\in H^{n-1}(\CS^{k-1}S(\xi))$ whose restriction to every fibre
$S^{n-1}$ is non-zero. The set of all $k$-flag structures on $\xi$ is denoted
by $\Fl^k(\xi)$.
}\end{definition}
\begin{proclaim}{{3.3.2. Theorem} (1) $\Fl^k(\xi)$ is not empty
if and only if $w_n(\xi)=\dots=w_{n-k+1}(\xi)=0$. In this case
$\Fl^k(\xi)$ is an affine space over $\bigoplus\limits_{i=1}^k H^{n-i}(X)$;

(2) if $l<k$, the inclusion homomorphism 
$H^{n-1}(\CS^{k-1}S(\xi))\longto H^{n-1}(\CS^{l-1}S(\xi))$
restricts to an affine map $\Fl^k(\xi)\longto\Fl^l(\xi)$;

(3) if $\Fl^k(\xi)$ is not empty, $\chi^k$ restricts to
an affine isomorphism $\Fl^k(\xi)\longto\prod\limits_{i=1}^k\St_{n-i}(\xi)$.
The following diagram commutes: ($l\le k$)
$$\begin{CD}
\Fl^k(\xi)@>\text{restriction}>>\Fl^l(\xi)\\
@VV\chi^k V@VV\chi^l V\\
\prod_{i=1}^k\St_{n-i}(\xi)
@>{\text{projection}}>>
\prod_{i=1}^l\St_{n-i}(\xi)
\end{CD}
$$
}\end{proclaim}
\begin{proof}{ Statements (1) and (2) follow from 3.1.2;
statement (3) follows from 3.2.5. 
}\end{proof}
\begin{definition}{{3.3.3. Definition} Denote by $p$ the projection
$\CS^{k-1}S(\xi)\longto\Rp^{k-1}\tm X$ and fix a subgroup (not necessary
graded) $G\subset\bigoplus\limits_{i=1}^k H_{n-i}(X)=H_{n-1}(\Rp^{k-1}\tm X)$.
A {\it partial $k$-flag structure\/} (defined over $G$) is a class
$\k\in H^{n-1}(\CS^{k-1}S(\xi))/\Ann(p_*^{-1}G)$ 
whose restriction to every fibre $S^{n-1}$ is non-zero.
The set of all partial $k$-flag structures
on $\xi$ is denoted by $\Fl^k(\xi,G)$.
(Remark: a $k$-flag structure can be thought of as a functional
$H_{n-1}(\CS^{k-1}S(\xi))\longto\Z$. A partial structure is a partial
functional defined on $p_*^{-1}G$. This explains the term.)
}\end{definition}
In the case of partial structures Theorem 3.3.2 should be
reworded as follows:
\begin{proclaim}{{3.3.4. Theorem} (1) $\Fl^k(\xi,G)$ is not empty
if and only if $w_i(\xi)=0$ for $i\ge n-k+1$. (Note that
this condition does not depend on $G$.) This set is an affine
space over $\Bigl[\bigoplus\limits_{i=1}^k H^{n-i}(X)\Bigr]\Big/\Ann G$;

(2) if $l\le k$, the inclusion homomorphism induces an affine map
$\Fl^k(\xi,G)\longto
\Fl^l\Bigl(\xi,G\cap\bigoplus\limits_{i=1}^l H^{n-i}(X)\Bigr)$.
}\end{proclaim}
%
%     The end of file 3.
%     File 4 of
% ``Stiefel orientations on a real algebraic variety''
%    by A.I.Degtyarev
%
\section{Flag structures on the fixed point set\\of an involution }
\subsection{Construction }
Let $X$ be an $(n+m)$-dimensional closed smooth manifold and $c$ be a smooth
involution on $X$. Fix some $c$-invariant $m$-dimensional closed smooth
submanifold $V\in X$ and denote by $F_V$ the intersection $V\cap \Fix c$,
by $i:F_V\into\Fix c$, \ $j:F_V\into V$ and $i_V:V\into X$ the inclusions,
and by $\nu_V$ the normal bundle of $V$ in $X$.
\begin{proclaim}{{4.1.1.Theorem} Let for some $k\ge 1$ class $(i_V)_*[V]$ belongs to
$\SF_{m+k}^{k+1}(X)$, or, equivalently, $e_m^1[V]=\dots=e_{m+k-1}^k[V]=0$.
Then there exists a natural $k$-flag structure $\k_k$ on $j^*\nu_V$ defined
over $i_*^{-1}\F_n^{k+1}(X)$. In particular, $w_p(j^*\nu_V)=0$ for $p>n-k$.
}\end{proclaim}
\begin{proof}{
Consider an equivariant tubular neighborhood $T_{k-1}$ of $\CS^{k-1}V$ in
$\CS^{k-1}X$ and denote by $\d T_{k-1}$ its boundary,
by $\pr:\d T_{k-1}\to \CS^{k-1}V$ the projection, and by
$\i:T_{k-1}\into \CS^{k-1}X$ the inclusion. The restriction
$\pr\left\vert\vphantom{(_(}\right.{}_{\pr^{-1}F_V}$ obviously coincides with
the sphere bundle associated with $\eta^{k-1}\o j^*\nu_V$, so the orientation
$\k_k$ that is to be constructed is a partial functional
$H_{n-1}(\d T_{k-1})\dashrightarrow\Z$, which is non-trivial on the fundamental
class of every fibre $S^{n-1}$ of $\pr$. We let $\k_k$ be the linking
coefficient map. More precisely, consider the following commutative diagram
with exact rows: ($D^{-1}$ denotes the Poincaris duality isomorphism)
$$\begin{CD}
@.\bar\k_k&\longmapsto&[\CS^{k-1}V]\\
&&\cap&&\cap\\
H_{m+k}(\CS^{k-1}X)@>\rel>>H_{m+k}(\CS^{k-1}X,T_{k-1})@>\d>>
   H_{m+k-1}(T_{k-1})@>\i^*>>\\
@VV D^{-1} V@VV D^{-1} V@VV D^{-1} V\\
H^{n-1}(\CS^{k-1}X)@>>>H^{n-1}(\CS^{k-1}X\setminus T_{k-1})@>\cd>>
   H^n(T_{k-1},\d T_{k-1})@>>>
\end{CD}
$$
Due to 2.5.5 the hypothesis of the theorem implies that $\i_*[\CS^{k-1}V]=0$.
Hence there exists an element $\bar\k_k\in H_{m+k}(\CS^{k-1}X, T_{k-1})$
such that $\d\bar\k_k=[\CS^{k-1}V]$; this element is unique up to $\Im\rel$.
Consider $D^{-1}\bar\k_k$ as an element of
$\Hom(H_{n-1}(\CS^{k-1}X\setminus T_{k-1}),\Z)$ and define $\k_k$ as the
composed homomorphism
$$\begin{CD}
H_{n-1}(\pr^{-1}F_V)@>\inc>>H_{n-1}(\CS^{k-1}X\setminus T_{k-1})
@>D^{-1}\bar\k_k>>\Z\end{CD}
$$
restricted to
$$
\Ker[\inc_*:H_{n-1}(\pr^{-1}F_V)\to H_{n-1}(\CS^{k-1}X)].
$$

Since $\cd D^{-1}\bar\k_k=D^{-1}[\CS^{k-1}V]$ is the Thom class of $\pr$,
the restriction of $\k_k$ to every fibre is non-zero, and $\k_k$ is a
$k$-flag structure. The domain of $\k_k$ is the pull back of
$\Ker[\i_*:H_{n-1}(\CS^{k-1}F_V)\to H_{n-1}(\CS^{k-1}X)]$;
due to 2.5.5 the latter group coincided with $\F_n^{k+1}$. 
}\end{proof}
4.1.3. Let $(i_V)_*[V]\in\SF_{m+k}^{k+1}$. Then due to 2.5.3 \
$(i_V)_*[V]\in\SF_{m+l}^{l+1}$ for any $l\le k$. This gives rise to series of
natural partial $l$-flag structures $\k_l$ on $j^*\nu_V$, \ $1\le l\le k$,
defined over $\F_n^{l+1}$. On the other hand, due to 2.3.4 $\k_k$ can be
restricted to an $l$-flag structure $\k_l^\prime$ defined
over subgroup $\F_n^{k+1}\cap\bigoplus\limits_{i=1}^l H_{n-1}(X)$,
which contains $\F_n^{l+1}$ (see 2.5.3).
\begin{proclaim}{ {4.1.4. Proposition} The restriction of
$\k_k$ to $\F_n^{l+1}$ coincides with $\k_l$.
}\end{proclaim}
\begin{proof}
Denote by $\i:(\CS^{l-1}X, T_{l-1})\into(\CS^{k-1}X,T_{k-1})$ the inclusion
and consider the following diagram
$$\begin{CD}
H_{m+k}(\CS^{k-1}X,T_{k-1})@>\d>>H_{m+k-1}(T_{k-1})\\
@VV\i^!V@VV\i^!V\\
H_{m+l}(\CS^{l-1}X,T_{l-1})@>\d>>H_{m+l-1}(T_{l-1})
\end{CD}
$$
Since $\i^![\CS^{k-1}V]=[\CS^{l-1}V]$, we can choose $\bar\k_l=\i^!\bar\k_k$.
Then $D^{-1}\bar\k_l=in^*D^{-1}\bar\k_k$, that shows that $\k_l$ coincides with
the restriction of $\k_k$. 
\end{proof}
\subsection{Geometric description }

4.2.1. The definition of map $\chi^k$ (3.2.4), filtration $\F_q^r$ and
homomorphisms $e_q^r$, and the geometric construction of the linking
coefficient map yield the following clear geometric description of flag
structures $\k_k$: (For simplicity we replace $\F_n^{k+1}$ with its maximal
graded subgroup. This allows to consider orientations of different dimensions
separately.)

Suppose that the total space of the sphere bundle $S(\nu_V)$ is embedded as
the boundary of an equivariant tubular neighborhood of $V$ in $X$.
Let $\sg:P\to V_{n,k}(j^*\nu_V)$ be an $(n-k)$- dimensional cycle (i.e. a
singular polyhedron, a singular manifold, a cell cycle, etc.) Consider the
composed map 
$$\begin{CD}\tilde\sg_i:P\tm D_+^i\into P\tm S^{k-1}@>\tilde\sg>>
S(j^*\nu_V)\into X\end{CD} $$ 
as a chain in $X$ (of the same nature as $\sg$), and
suppose that there exist some $j$-dimensional chains $y_j$ in$X$, \
$n-k+1\le j\le n$, such that $\d y_{n-k+1}=\tilde\sg_0$, and
$\d y_j=(1+c_\s)y_{j-1}+\tilde\sg_{j+k-n-1}$ for $j>n-k+1$. Suppose that
$y_n$ is transversal to $V$. Then $\k_k(\sg)=\#(y_n\cap F)\mod2$.
\begin{remark}{{Remark} It is this construction that
was originally proposed by O.Viro \cite{8}.
}\end{remark}
\subsection{Projective complete intersections
}

4.4.1. Let $A$ be an $n$-dimensional non-singular real algebraic variety.
Denote by $\CA$ the set of complex points of $A$, by $\RA$ the real part of
$A$, and by $\conj$ the involution of complex conjugation on $\CA$.
Multiplication by $\sqrt{-1}$ induces an isomorphism between the tangent
bundle $\tau_{\RA}$ of $\RA$ and its normal bundle $\nu_{\RA}$ in $\CA$.
Since $\RA=\Fix\conj$, Theorem 4.1.1 yields:
\begin{proclaim}{{4.4.2. Theorem} If $[\RA]\in\F_{n+k}^{k+1}(\CA)$, there exists a
natural partial $k$-flag structure on $\tau_{\RA}$ defined over
$\F_n^{k+1}(\CA)$. In particular, $w_p(\RA)=0$ for $p>n-k$.
}\end{proclaim}

4.4.3. Let $A$ be a projective complete intersection. Denote by $\Ch$ (resp,
$\Rh$) the complex (resp,real) hyperplane section class in $H^2(\CA,\mathbb Z)$
(resp, $H^1(\RA,\Z)$), and by $\tR$ the double cover of $\RA$ corresponding
to $\Rh$.
\begin{proclaim}{ {4.4.4. Theorem} Let $[\RA]$ is either equal to zero or dual to
$\Ch^{n/2}\mod2$ in $H_n(\CA)$. Then for every $k$, \ $0\le k\le n-1$, \ $\tR$
is naturally supplied with a reduced $k$-dimensional Stiefel orientation.
}\end{proclaim}
\begin{remark}{ {Remark} In particular, 4.4.4 implies that $w_1(\tR)=\dots=w_n(\tR)=0$.
Note, however, that this holds for every projective complete intersection,
no matter whether $[\RA]$ vanishes or not, since $\tR$ is a complete
intersection in sphere $S^N=\widetilde{\Rp^N}$.
}\end{remark}
\begin{proof}{ Consider the following diagram
$$\require{AMScd}\begin{CD}
\tR@>\i>>\pC^{-1}\RA@>\text{inclusion}>>\tC\\
&{}_{\pR}\searrow&@VVV@VV\pC V\\
@.\RA@>\text{inclusion}>>\CA
\end{CD}
$$
where $\pC:\tC\to\CA$ is the $S^1$-bundle corresponding to $\Ch$. \ $\tC$ can
be constructed as the pull back of $\CA$ under the standard projection
$S^{2N+1}\to\Cp^N\supset\CA$. The complex conjugation on
$S^{2N+1}\subset\C^{N+1}$ restricts to involution $\tconj$ on $\CA$ such that
$\tconj$ covers $\conj$, \ $\pC^{-1}\RA$ is $\tconj$-invariant,
and $\RA=\Fix\tconj$.

Class $[\pC^{-1}\RA]=\pC^![\RA]$ vanishes in $H_{n+1}(\CA)$. Since $H^*(\CA)$
is generated by powers of $\Ch$ up to dimension $n-1$, \
$H_i(\tC)=H^{2n+1-i}(\tC)=0$ for $n+2\le i\le 2n$, and all maps $e_q^r$ are
trivial for $n+2\le q\le 2n$. Hence $[\pC^{-1}\RA]\in\SF_{2n+1}^{n+1}(\tC)$,
and Theorem 4.1.1 applied to $V=\pC^{-1}\RA$ yields a partial $k$-flag
structure $\k_n$ on the bundle 
$\i^*\nu_V=\pR^*\nu_{\RA}\cong\pR^*\tau_{\RA}=\tau_{\tR}$. This structure is
defined over subgroup $\F_{n+1}^{n+1}$, which coincides with
$\widetilde H_*(\tR)$ since $\widetilde H_i(\tC)=0$ for $i\le n$.
Due to Theorem 3.3.2 \ $\chi^n\k_n$ is a collection of reduced $k$-dimensional
Stiefel orientations, $0\le k\le n-1$. 
}\end{proof}
%
\begin{thebibliography}{2}

\bibitem[1]{1}
 G. E. Bredon, {\it Introduction to compact transformation groups,}
 Academic Press,  New York, London
 1972

\bibitem[2]{2}
 A. Degtyarev,
{\it Cohomology approach to structures on $G$-bundles}
to appear
\bibitem[3]{3}A.Degtyarev,
{\it Screwed Steenrod squares and some applications}
to appear
\bibitem[4]{4}
 W. I. Hsiang,
{\it Cohomology theory of topological transformation groups,}
 Springer-Verlag,  Berlin, Heidelberg, New York
1975
\bibitem[5]{5}
I. Kalinin
{\it Cohomology characteristics of real algebraic hypersurfaces,} 
Algebra i Analis
{\bf3} (1991) Russian
\bibitem[6]{6}
V. A. Rokhlin,
{\it Complex orientations of real algebraic curves,}
 Funkz. Analis 
( Russian)
{\bf 8}  (1974)  71--75
\bibitem[7]{7}
 N. E. Steenrod, D. B. A. Epstein
{\it Cohomology operations,}
Princeton University Press, Princeton (1962)
\bibitem[8]{8}
 O. Ya. Viro,
{\it Progress in the topology of real algebraic varieties
over the last six years,}
 Russian Math. Surveys
{\bf41}  (1986) 55--82
\end{thebibliography}
\end{document}
