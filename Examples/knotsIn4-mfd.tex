\documentclass{article}
\usepackage{textcomp,verbatim,amssymb,amsmath,array}
\def\<{\textless}
\def\>{\textgreater}
\usepackage{vo}
\pagestyle{empty}
\begin{document}


$\newcommand\Fitt{\operatorname{Fitt}}
\newcommand{\Z}{\mathbb Z}
\newcommand{\p}{\partial}
\newcommand{\inj}{\operatorname{in}}
\newcommand{\codim}{\operatorname{codim}}
\newcommand{\sminus}{\smallsetminus}$ 

\spnewtheorem{Th}{Theorem}[section]{\bf}{\it}



\title{Space of smooth 1-knots in a 4-manifold:\\ Is its algebraic topology
sensitive to smooth structures?}
\author{Oleg Viro}
\date{}
\maketitle

\section{Introduction}\label{s1}
\subsection*{ Diffeomorphic vs. homeomorphic in high dimensions}
There are smooth manifolds, which are homeomorphic, but not diffeomorphic.
It happens to manifolds of dimension greater than 3. 
This phenomenon was discovered by Milnor \cite{Mi1} for the 7-dimensional
sphere $S^7$ in the fifties. A technique for classification of 
smooth 
structures on a manifold of dimension greater than 4 was developed in the 
sixties. 
Homeomorphic, but not diffeomorphic manifolds were found in all 
dimensions $>4$, see \cite{KM}, \cite{Ba}, \cite{S}. 
A substantial part of the technique used in these works is not 
applicable in dimension 4. \smallskip


\subsection*{Closed simply connected 4-manifolds} 
In the eighties Freedman \cite{F} gave a topological classification 
of closed simply connected 4-manifolds and Donaldson \cite{D} proved that 
some smooth closed simply connected 4-manifolds, which are homeomorphic 
according to Freedman's results, are not diffeomorphic. 

For smooth closed simply connected 4-manifolds, the only 
homotopy invariant
is the intersection form in the second homology with integer coefficients:
two manifolds of this kind are homotopy equivalent iff their intersection
forms are isomorphic (Whitehead \cite{Wh}, Pontryagin \cite{Po}). 

As Freedman \cite{F} proved, for manifolds of this type homeomorphism 
is equivalent to homotopy equivalence. Thus, two smooth closed simply connected 
4-manifolds are homeomorphic iff their intersection forms are isomorphic. 

In contrast, on many closed simply connected 4-manifolds there are 
infinitely many smooth structures. Of course, the number of smooth structures 
on a closed 4-manifold cannot be uncountable (on a non-compact 4-manifold it 
can, as is the case for $\mathbb R^4$). There is no conjectural classification 
scheme
for smooth structures on any closed 4-manifold. Stern entitled his survey  
paper \cite{Stern} by a question ``Will we ever classify simply-connected 
smooth 4-manifolds?''. In the introduction to his paper he wrote: ``The 
subject is rich in examples that demonstrate a wide variety of disparate 
phenomena. Yet it is precisely this richness which, at the time of these 
lectures, gives us little hope to even conjecture a classification scheme.'' 


\subsection{Invariants distinguishing smooth structures}
Proving that homeomorphic manifolds of dimension 4 are not diffeomorphic 
required absolutely new tools. All the invariants, 
that have been used so far for proving that some homeomorphic
simply connected closed smooth 4-manifolds are not diffeomorphic, 
are deeply rooted in analysis. They are based on counting solutions to some 
nonlinear partial differential equations. 
Most of the results on non-existence of 
diffeomorphisms between homeomorphic simply connected 4-manifolds have been 
obtained 
using the Donaldson and Seiberg-Witten invariants.

It is a long term challenge for topologists to find invariants which  
would distinguish smooth structures on a 4-manifold, but would be independent
on analytic tools. This is not just an aesthetic issue: the analytic 
tools are not convenient in some situation. 

For example, in dimension 4 smooth structures are closely related to 
PL-structures (piecewise linear structures). In any dimension a
smooth structure on a manifold determines on it a specific PL-structure 
uniquely defined up to PL-homeomorphism.
In dimensions $\le 6$ any PL-structure can be obtained in this way from a 
smooth structure and the smooth structure is unique up to diffeomorphism,
see \cite{S}. 
Hence any invariant of a smooth structure on a 4-manifold 
depends only on the PL-structure. However,  in order to prove that 
two homeomorphic PL-manifolds of dimension 4 are
not PL-homeomorphic, now one has to equip them with smooth 
structures, and then calculate the invariants. 
The techniques for calculation are not easy, especially if 
the smooth structure is not defined naturally (say, as the underlying 
structure on a complex surface), 
but just obtained by smoothing of a PL-structure. 

\subsection{Combinatorial invariants}
It is worth mentioning a partial success of efforts towards eliminating 
of analysis.
An invariant of smooth 4-manifolds, the Ozsv\'ath-Szab\'o 
mixed invariant \cite{OS}, conjecturally coinciding with the Seiberg-Witten 
invariant, has been redefined \cite{MOT} in purely 
combinatorial terms. However, on one hand, it is difficult to calculate even 
for comparatively simple 4-manifolds, so it not an effective tool for 
distinguishing smooth structures, on the other hand, its combinatorial 
description has not 
clarified its nature,  has not related it to the rest of topology.

All the known invariants distinguishing smooth structures on closed simply 
connected 4-manifold $X$ require non-trivial intersection form on $H_2(X)$. 
In particular, for the 
4-sphere these invariants give nothing, they cannot help in disproving of the 
4-dimensional differential Poincar\'e conjecture, according to which any closed 
4-dimensional manifold homeomorphic to $S^4$ is diffeomorphic to $S^4$. 

Thus, a development of new approaches to building invariants of smooth 
structures is desirable for several good reasons.

\section{Seiberg-Witten versus Alexander}\label{s2}
One of the most powerful invariants, which
distinguish smooth structures on a 4-manifold, is the Seiberg-Witten
invariant. It can be presented in several forms. In particular,
the Seiberg-Witten invariant of a smooth closed 
simply-connected 4-manifold $X$ can be identified with an element ${SW}_X$ 
of $\Z[H_2(X)]$, the integer group algebra of the {\em second\/} 
homology group $H_2(X)$.

This identification makes the Seiberg-Witten invariant
resembling the Alexander polynomial of a link. 
For a classical link $L\subset S^3$, the Alexander polynomial 
is an element of 
$\Z[H_1(S^3\sminus L)]$, the integer group algebra of the {\em first\/} 
homology group. 

The Alexander polynomial can be defined in many ways. In particular, 
it admits purely topological definitions.

The similarity between Seiberg-Witten invariant and Alexander polynomial gives 
a hope to find either a definition of Seiberg-Witten 
invariant free of heavy analysis, or, at least, to invent 
similar invariants that solve the same problems, but are defined and 
calculated in ways more traditional for topology.
This hope is supported by a relation between the Seiberg-Witten invariant and 
the Alexander polynomial discovered 
by Fintushel and Stern in 1996 \cite{FS}:

\section{Fintushel-Stern knot surgery of a 4-manifold}\label{s3}
Let $X$ be any simply connected closed smooth 4-manifold. 
Suppose that $X$ contains a smoothly embedded torus $T$ with simply 
connected complement $X\sminus T$ and with zero
self-intersection number $T\circ T$.  Since $T\circ T=0$,
the normal bundle of $T$ is trivial, and a tubular neighborhood of $T$
can be identified with  $T\times D^2$. 

A {\em knot surgery\/} on $T$ takes away from $X$ a tubular neighborhood 
$T\times D^2$ of $T$  and attaches $S^1\times(S^3\sminus N_K)$ instead,  
where $N_K$ is an open tubular neighborhood of a smooth knot $K\subset S^3$.
Observe that the boundary of $S^3\sminus N_K$ is diffeomorphic to the 2-torus
$S^1\times S^1$, and hence the boundary of the inserted
 piece $S^1\times(S^3\sminus N_K)$ is diffeomorphic to
the 3-torus $S^1\times S^1\times S^1$ as well as the boundary 
$T\times\p D^2$ of the piece removed. The attaching is performed by a 
diffeomorphism $S^1\times\p(S^3\sminus N_K)\to T\times\p D^2$ which maps 
$\{pt\}\times l$, where $l$ is a longitude, 
i.e., a circle bounding in $ S^3\sminus K$, to a fiber $\{pt\}\times\p D^2$. 
This requirement on the attaching 
map does not determine the map up to diffeotopy, and hence does not 
necessarily determine
$X_K$ up to diffeomorphism. However, by the following theorem, under 
some assumptions, all the 
manifolds obtained from the same $(X,T)$ and $K\subset S^3$ by knot 
surgeries have the same Seiberg-Witten invariant.

\begin{Th}[{\bf Fintushel-Stern \cite{FS}}]\label{ThFS} Let $X$ be a 
simply connected closed smooth 4-manifold with $b_+(X)>1$, let $X$
contain  a smoothly embedded torus $T$ with $T\circ T=0$ and simply
connected complement $X\sminus T$. Let  $K$ be a knot in $S^3$. Then the
result $X_K$ of a knot surgery is
homeomorphic to $X$ and 
$$
{SW}_{X_K}={SW}_X\cdot \Delta_K(t^2),
$$
 where $t\in H_2(X)$ is the homology class realized by $T$ and
$\Delta_K$ is the symmetrized Alexander polynomial of $K$.
\end{Th}

Present definitions of the Seiberg-Witten invariant are not applicable to 
$(S^3\sminus K)\times S^1$ or $(S^3\sminus N_K)\times S^1$.
So, one cannot speak about ${SW}_{(S^3\sminus K)\times S^1}$.
However, if the ${SW}$ was extended to this setup and satisfied 
a reasonable additivity property, Fintushel-Stern Theorem would  
suggests that ${SW}_{(S^3\sminus K)\times S^1}$ should be $\Delta_K$.

\section{Is ${SW}_X$ the Alexander polynomial of
something?}\label{s4} The Alexander polynomial of a classical knot $K$ is
the order of $\Z[H_1(S^3\sminus K)]$-module $H_1(\widetilde{S^3\sminus
K})$, where $\widetilde{S^3\sminus K}\to S^3\sminus K$ is the infinite
cyclic covering. The automorphism group of this covering is $H_1(S^3\sminus
K)=\Z$, it acts
in the homology group $H_1(\widetilde{S^3\sminus K})$, turning it into a
$\Z[\Z]=\Z[t,t^{-1}]$-module. 

This module is called the {\em Alexander module\/} of $K$. It is  
finitely generated and admits a square matrix of relations.
The determinant of this matrix is the Alexander polynomial. 
It is defined up to multiplication by units 
of $\Z[\Z]$ (that is by monomials $\pm t^k$).    

According to this construction, the Alexander polynomial happens to belong to
$\Z[H_1(S^3\sminus K)]$. The Seiberg-Witten invariant of $X$ belongs to 
$\Z[H_2(X)]$. What could be a space $Y$ such that $H_1(Y)=H_2(X)$?

\section{The loop space $\Omega X$?}\label{s5}
There is an obvious candidate for such $Y$, the loop space $\Omega X$ of $X$.
Indeed, $\pi_i(\Omega X)=\pi_{i+1}(X)$, and, in particular, 
$\pi_1(\Omega X)=\pi_2(X)$; in the case of simply connected $X$, 
 $\pi_2(X)$ is isomorphic to $H_2(X)$ by the Hurewicz theorem. 
Thus $\pi_1(\Omega X)=H_2(X)$. Therefore $\pi_1(\Omega X)$ is commutative,
and hence  $H_1(\Omega X)=\pi_1(\Omega X)=H_2(X)$.

However, the loop space per se cannot do the job:\begin{itemize}
\item
First, the homotopy type of $\Omega X$ depends only on the homotopy type of
$X$. Therefore homotopy invariants of $\Omega X$ cannot distinguish 
smooth structures on $X$.
\item
Second, $\Omega X$ is an $H$-space. Therefore $\pi_1(\Omega X)$ acts on
$H_*(\widetilde{\Omega X})$ trivially.
\end{itemize}

\section{ The knot space?}\label{s6}
How to improve $\Omega X$? 

First, let us make it closer to the smooth structure of $X$. The loop space 
$\Omega X$ contains the space $\Omega_{Diff}X$ of differentiable loops 
$S^1\to X$ that have at the base point fixed non-vanishing differential. 
The replacement of $\Omega X$ by $\Omega_{Diff}X$ does not change the 
homotopy type: well-known approximation theorems imply that 
$\Omega_{Diff}X$ is a deformation retract of $\Omega X$. 

Let $KX\subset \Omega_{Diff}X$ be the subspace which consists of loops that are 
smooth embeddings. Denote $\Omega_{Diff}X\sminus KX$ by $DX$.
Observe that $\codim_{\Omega_{Diff}X}DX=2$,  hence the 
inclusion homomorphism
$\pi_1(KX)\to\pi_1(\Omega_{Diff} X)=\pi_1(\Omega X)$ is onto. 

Obvious suggestion: 
consider the covering $\widetilde{KX}\to KX$ induced by the universal
covering  $\widetilde{\Omega X}\to \Omega X$. The automorphism group 
of the covering $\widetilde{\Omega X}\to \Omega X$ is $H_1(\Omega
X)=H_2(X)$. Therefore $H_*(\widetilde{\Omega X})$ and 
$H_*(\widetilde{KX})$ are modules over % the ring 
$\Z[H_1(\Omega X)]=\Z[H_2(X)]$.

The action of $H_1(\Omega X)$ in $H_*(\widetilde{\Omega X})$ is trivial,
since $\Omega X$ is an $H$-space, while $KX$ is not an $H$-space and 
$H_*(\widetilde{KX})$ may be an interesting $\Z[H_2(X)]$-module. 
It has a broad range of invariants belonging to $\Z[H_2(X)]$
similar to the Alexander polynomial. Indeed, if $X$ is simply connected,
then $H_2(X)$ is a free abelian group of finite rank $r$, and $\Z[H_2(X)]$ is
isomorphic to a ring $\Lambda_r=\Z[t_1,t_1^{-1},t_2,t_2^{-1},\dots 
t_r,t_r^{-1}]$
of Laurent polynomial in $r$ variables with integer
coefficients, the same ring as in the situation of the Alexander module of a
classical link. 


A finitely generated module $M$ over $\Lambda_r$ gives rise to a
filtration  of $\Z[H_2(X)]$
$$\Fitt_0(M)\subset\Fitt_1(M)\subset\dots\subset\Z[H_2(X)]$$ 
by {\em Fitting ideals\/}.  The $i$th Fitting ideal $\Fitt_i(M)$ is 
generated by the minors (determinants of submatrices) of order $r-i$ 
of the matrix of defining relations for $M$   
In the topological literature Fitting ideals are called also {\em
elementary ideals\/}. A generator of the minimal principal ideal 
containing $\Fitt_i(M)$ is denoted by $\Delta_i(M)$. 

In the context of link theory, when $\Lambda_r=H_1(S^3\sminus L)$, where 
$L$ is an $r$-component link and $M=H_1(\widetilde{S^3\sminus L})$, the
Laurent polynomial $\Delta_i(M)$ is called the $i$th Alexander polynomial
of $L$. The $0$th Alexander polynomial is one of the oldest link
invariants. It was introduced by Alexander \cite{Al}. 

Similarly, for a smooth simply connected closed 4-manifold $X$, each
$H_i(\widetilde{KX})$, as a module over $\Z[H_2(X)]$, gives rise to 
a sequence of Laurent polynomials. They resemble the Seiberg-Witten 
invariant. At least, they belong to the same $\Z[H_2(X)]$.
The modules $H_i(\widetilde{KX})$ are also invariants of $X$.\medskip


\<p\>
{\bf The problem:} {\em Does there exist homeomorphic smooth simply 
connected closed 4-manifolds $X_1$, $X_2$ such that  $H_*(\widetilde{KX_1})$
and  $H_*(\widetilde{KX_2})$ are not isomorphic?\/}

\section{Is this plausible?}\label{s7}
The author discussed the problem with many leading specialists in the field 
and asked them this question. The answers spread over a broad spectrum. Only 
one expert expressed a definitely negative opinion. 
On the other end, there were also very enthusiastic reactions. 

The most convincing argument is that for different smooth structures on the
same topological manifold $X$ a topologically observable difference between
smooth structures was the minimal genus of a smoothly embedded surface 
realizing  an element of $H_2(X)$. For any smooth structure each class can be 
realized
by an immersed sphere with transverse self-intersection (by the Hurewicz 
theorem and transversality), but the minimal number of
double point of such immersion depends on the structure. It is greater than 
or equal to the minimal genus of a smoothly embedded surface realizing the
class.  

An immersion $f:S^2\to X$ can be used to produce a loop in the space of
loops $\Omega_{Diff} X$. A loop in $\Omega_{Diff}S^2$ sweeping the whole $S^2$
composed with $f$ gives a loop in $\Omega_{Diff} X$. Generically this gives a 
loop
in $KX$, because under a generic choice of loop in $\Omega_{Diff} S^2$, points 
in 
the preimage of a double point of $f$ are passed at different moments. 
However, in families of loops realizing elements of high-dimensional 
homology groups $H_k(\Omega_{Diff} S^2)=\Z$ these pairs of points appear on some
of the loops. Such a family does not realize a homology class in $KX$ or
$\widetilde{KX}$.    


\section{Vassiliev invariants for 4-manifolds?}\label{s8}
How to calculate $H_i(\widetilde{KX})$? One may apply Vassiliev's idea
\cite{Va}, which 
led to discovery of the Vassiliev knot invariants: start the calculation 
with a study of the dual cohomology, the cohomology of the space of singular
knots. The space of singular knots has a rich geometric structure. 


The space $DX$ 
consists of differentiable loops that are not embeddings. It fits to 
the collection of discriminant hypersurfaces studied by Vassiliev \cite{Va}.
The universal covering $\widetilde{\Omega X}\to \Omega X$ defines 
a covering $\widetilde{DX}\to DX$.

Resolve singularities of the discriminant $DX$, as Vassiliev
did. This gives rise to a filtration in $H^*(\widetilde{DX})$.
The first terms of these filtration are easier to calculate than the whole
cohomology group.


At first glance, the situation is much more complicated than 
in the original setup in the theory of Vassiliev knot invariants. 
Let us examine the extra difficulties.

First, instead of the space of singular knots, we have to deal with its
covering space. What does happen, when we pass from knots to
points of a covering space? 

In general, if $p:X\to B$ is a covering, $b_0\in B$, $x_0\in X$ are points 
such that
$p(x_0)=b_0$ and $X$ is path connected, then a point $x\in X$ is uniquely
determined by its image $p(x)\in B$, a path $s:I\to B$ such that $s(0)=b_0$,
and its covering path $\widetilde s:I\to X$ starts at $x_0$ and finishes
at $x$. The path $s$ is defined up to path homotopy and multiplying by loops 
covered by loops in $X$.

In particular, a point of $\widetilde{\Omega X}$ (no matter if it belongs
to $\widetilde{KX}$ or $\widetilde{DX}$) is defined by a loop
$u:S^1\to X$ and a continuous map $f:D^2\to X$ with $f|_{S^1}=u$
considered up to homotopy which is fixed on $S^1$. 
The action of $H_2(X)$ in $\widetilde{\Omega X}$ is realized in this model
as addition to the homotopy class of $f$ 
homotopy classes of  maps $S^2\to X$ realizing elements of $H_2(X)$. 

Second, the space of all loops in the Vassiliev setup is contractible.
(Recall that loops there have the base point and the tangent vector at the
base point fixed.) This allowed to apply the Alexander duality between the
homology of the discriminant and cohomology of its complement, the space of
knots. This was done in a finite dimensional (say, polynomial)
approximation of the loop space.

Here the loop space is not contractible. Therefore, the Alexander duality  
is not applicable. However, it may be replaced by the Alexander-Poincar\'e
duality between the homology of $\widetilde{KX}$ and a relative cohomology, 
the cohomology of $(\widetilde{\Omega_{Diff} X},\widetilde{DX})$.

\section{An alternative approach: string topology}\label{s9}

The homology of loop space $\Omega_{Diff}X$ and its universal covering 
accommodate a rich structure of the string topology operations \cite{CS}.
Geometrically, one can expect that this structure incorporates the same 
information as the 
natural filtration of the discriminant and the Alexander-Poincar\'e duality.
Apparently the connection have not been investigated. Nonetheless, it would
be natural to expect that the string topology is as strong as the invariants
discussed above. 

\<p\> 
{\bf Problem.\/} {\it Is the string topology sensitive to smooth structures
on 4-manifolds?} 

\<p\>It would be interesting also to find direct relations between the Vassiliev
invariants theory and the string topology in the setup of classical knot 
theory.  


\begin{thebibliography}{99}

\bibitem[{Alexander}{1928}]{Al}  J.W. Alexander, {\it Topological invariants of knots and
links\/}, Trans. Amer. Math. Soc. 30 (2) (1928) 275-306.

\bibitem[{Barden}{1987}]{Ba} D. Barden, {\it Simply connected five-manifolds,\/} 
Ann. of Math. (2) 82 (1965), 365-385.

\bibitem[{Donaldson}{1987}]{D}  S. Donaldson, {\it Irrationality and the h-cobordism 
conjecture,\/} J. Differential Geom. 26 (1) (1987) 141-168.

\bibitem[{Donaldson}{1983}]{D1} S. Donaldson, {\it An application of gauge theory to 
4-dimensional topology,\/} J. Diff. Geom., 18 (1983), 279-315.

\bibitem[{Chas and Sullivan}{1999}]{CS} Moira Chas and Dennis Sullivan. {\it String topology.\/} 
Preprint arXiv:math/9911159.

\bibitem[{Freedman}{1982}]{F}  Michael Hartley Freedman, {\it The topology of four-dimensional
manifolds\/}, Journal of Differential Geometry 17 (3)  (1982), 357-453.

\bibitem[{Fintushel and Stern}{1998}]{FS}  R. Fintushel and R.Stern, {\it Knots, links, and
4-manifolds,\/} Invent. Math. {\bf 134}, no. 2 (1998), 363-400.

\bibitem[{Milnor}{1956}]{Mi1} John W. Milnor, {\it On manifolds homeomorphic to the
7-sphere,\/}
Ann. of Math., 64 (1956), 399-405.

\bibitem[{Manolescu, Ozsv\'{a}th and Thurston }{2009}]{MOT} Ciprian Manolescu, Peter s. Ozsv\'ath, and Dylan P. Thurston,
{\it Grid diagrams and Heegaard Floer invariants\/}, arXiv:0910.0078 [math.GT].


\bibitem[{Kervaire and Milnor}{1963}]{KM}  Michel Kervaire and John W. Milnor, {\it Groups of homotopy
spheres,\/} Ann. of Math., 77 (1963), 504-537.

\bibitem[{Ozsv\'{a}th and Z. Szab\'{o}}]{OS} P. S. Ozsv\'{a}th and Z.
Szab\'{o},  {\it Holomorphic triangles and 
invariants for smooth four-manifolds,\/}   Adv. Math., 202(2006), no. 2, 
326-400.

\bibitem[{Pontryagin}{1949}]{Po} L.S.Pontryagin, {\it On classification of four-dimensional
manifolds,\/} Uspekhi Mat. Nauk, 4:4 (1949), 157-158 (Russian).

\bibitem[{Siebenmann}{1970}]{S} L.C. Siebenmann, {\it Topological manifolds,\/} 
Proceedings of the International Congress of Mathematicians
Nice, September, 1970, Gauthier-Villars,
1971, Vol. 2, 133-163.

\bibitem[{Stern}{2006}]{Stern} Ronald J. Stern, {\it `Will we ever classify simply-connected 
smooth 4-manifolds?\/}, Clay Mathematics Proceedings, Volume 5, 2006, 225-239.

\bibitem[{Vassiliev}{1994}]{Va} V.A. Vassiliev, {\it Complements of Discriminants of Smooth
Maps: Topology and Applications\/}  Translations of Mathematical Monographs 
Series, 98,  American Mathematical Society, second edition, 1994. 

\bibitem[{Whitehead}{1949}]{Wh} J.H.C. Whitehead, {\it On simply connected 4-dimensional
polyhedra,\/} Comm. Math. Helv., 22 (1949), 48-92.

\end{thebibliography}
\end{document}
